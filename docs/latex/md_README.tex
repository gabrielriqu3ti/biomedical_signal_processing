This project encapsulates some small projects developed in the course \href{https://uspdigital.usp.br/jupiterweb/obterDisciplina?nomdis=&sgldis=PTC3456}{\tt P\+T\+C3456 -\/ Biomedical Signals Processing}. Each on aims to give some elucidation and simulate the human body and its signals.

The available projects are\+:


\begin{DoxyItemize}
\item \href{https://github.com/gabrielriqu3ti/biomedical_signal_processing/wiki/Nernst-Equation}{\tt Nernst Equation}
\item \href{https://github.com/gabrielriqu3ti/biomedical_signal_processing/wiki/Goldman-Equation}{\tt Goldman Equation}
\end{DoxyItemize}

You can run an application of each project from the main application by running\+:


\begin{DoxyCode}
python main\_app.py
\end{DoxyCode}


\subsection*{Nernst Equation}

The {\bfseries Nernst Equation} allows us to calculate the equilibrium potential for a given ion separated by a phospholipid membrane with ion channels selectively permeable to that ion and is given by\+:



where\+:


\begin{DoxyItemize}
\item E\+\_\+ion = ionic equilibrium potential
\item R = gas constant
\item T = absolute temperature
\item z = charge of that ion
\item F = Faraday\textquotesingle{}s constant
\item ln = natural logarithm
\item \mbox{[}ion\mbox{]}out = concentration of the ion outside the cell
\item \mbox{[}ion\mbox{]}in = concentration of the ion inside the cell
\end{DoxyItemize}

The {\bfseries equilibrium potential} is the electrical potential difference that exactly balances an {\bfseries ionic concentration gradient}.

\subsubsection*{Tools}

\paragraph*{The ionic equilibrium for a human neuron}

The equilibrium potential and the ionic concentration of ions in a human neuron are displayed by running the following code\+:


\begin{DoxyCode}
python human\_neuron\_equilibrium.py
\end{DoxyCode}


\paragraph*{App}

To calculate the equilibrium potential of an ion for a specific setting, run\+:


\begin{DoxyCode}
python nernst\_app.py
\end{DoxyCode}


\paragraph*{The ionic equilibrium for an exercise of the course P\+T\+C3456}

The equilibrium potential and the ionic concentration of ions in the exercise are displayed by running the following code\+:


\begin{DoxyCode}
python ex\_nernst\_equation.py
\end{DoxyCode}


\subsection*{Goldman Equation}

The {\bfseries Goldman Equation} allows us to calculate the resting membrane potential for a given set of monovalent ions separated by a phospholipid membrane with ion channels selectively permeable to these ions and is given by\+:




\begin{DoxyItemize}
\item E\+\_\+r = resting membrane potential
\item R = gas constant
\item T = absolute temperature
\item z = charge of that ion
\item F = Faraday\textquotesingle{}s constant
\item ln = natural logarithm
\item \mbox{[}cation\mbox{]}out = concentration of the monovalent cation outside the cell
\item \mbox{[}cation\mbox{]}in = concentration of the monovalent cation inside the cell
\item P\+\_\+cation = relative permeability of the cell membrane to the monovalent cation in relation to an ion
\item \mbox{[}anion\mbox{]}out = concentration of the monovalent anion outside the cell
\item \mbox{[}anion\mbox{]}in = concentration of the monovalent anion inside the cell
\item P\+\_\+anion = relative permeability of the cell membrane to the monovalent anion in relation to an ion
\end{DoxyItemize}

The {\bfseries resting membrane potential} is the electrical potential difference across an membrane not conducting {\bfseries action potentials}.

\subsubsection*{Tools}

\paragraph*{The resting potential for a human neuron}

The resting potential and the ionic concentration of ions in a human neuron are displayed by running the following code\+:


\begin{DoxyCode}
python human\_neuron\_equilibrium.py
\end{DoxyCode}


\paragraph*{App}

To calculate the equilibrium potential of an ion for a specific setting, run\+:


\begin{DoxyCode}
python goldman\_app.py
\end{DoxyCode}
 